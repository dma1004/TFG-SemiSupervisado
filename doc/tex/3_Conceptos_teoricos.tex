\capitulo{3}{Conceptos teóricos}


\section{Aprendizaje automático}

https://www.iebschool.com/blog/que-machine-learning-big-data/

Según \cite{intelligent:ml}, el aprendizaje automático (machine learning) es una
rama de la Inteligencia artificial como una técnica de análisis de datos que
enseña a las computadoras aprender de la \textbf{experiencia} (es decir, lo que
realizan los humanos). Para ello, el aprendizaje automático se nutre de gran
cantidad de datos (o los suficientes para el problema concreto) que son
procesador por ciertos algoritmos. Estos datos son ejemplos \cite{pascual:ml}
mediante los cuales, los algoritmos son capaces de generalizar comportamientos
que se encuentran ocultos. La característica principal de estos algoritmos es
que son capaces mejorar su rendimiento de forma automática en base a procesos de
entrenamiento y en las fases posteriores de explotación. Debido a su
propiedades, el aprendizaje automático se ha convertido en un campo de alta
importancia, aplicándose a multitud de campos como medicina, automoción, visión
artifical\ldots Los tipos de aprendizaje automático son: aprendizaje
supervisado, aprendizaje no supervisado, aprendizaje por refuero y aprendizaje
semi-supervisado. En la figura \ref{fig:taxonomia} se puede ver una
clasificación de aprendizaje automático (basada en \cite{neova:taxonomy}).

\subsection{Aprendizaje surpervisado}


\imagen{taxonomia}{clasificación de aprendizaje automático}{1}



\subsection{Aprendizaje no surpervisado}

\subsection{Aprendizaje semi-surpervisado}

